%-----------------------------------------------------------
% These packages are essential to produce the poster
%-----------------------------------------------------------
\usepackage[scale=1.24]{beamerposter}
\usepackage{graphicx,amsfonts}
\newcounter{figura}
\setcounter{figura}{1}

%-----------------------------------------------------------
% Custom commands that I use frequently
%-----------------------------------------------------------
\newcommand{\bb}[1]{\mathbb{#1}}
\newcommand{\cl}[1]{\mathcal{#1}}
\newcommand{\fA}{\mathfrak{A}}
\newcommand{\fB}{\mathfrak{B}}
\newcommand{\Tr}{{\rm Tr}}
\newtheorem{thm}{Theorem}
\newcommand{\captione}[2]{\begin{minipage}[l]{#1}\begin{center} \textit{Figure \thefigura.}\  #2\end{center}\end{minipage}\addtocounter{figura}{1}}
\newcommand{\spacer}{\begin{column}{\sepwid}\end{column}}
\newcommand{\toplogo}[2]{\newcommand{\posterlogo}{\includegraphics[width=#1]{mypics/#2}}}
\newcommand{\extrainfo}[1]{\newcommand{\inforight}{#1}}
\newcommand{\moreinfo}[1]{\newcommand{\inforightother}{#1}}
\newcommand{\conjetura}[2]{\\[-8mm]\begin{minipage}[l]{45cm}\emph{#1}\end{minipage}& \,\hspace{2cm} \emph{#2}\hspace{2cm}\,\\[-6mm]  \\ \hline}


%-------------------------------------------------------------------------------------------------------------------------------
% These commands will help me to build new blocks, do not modify!
%-------------------------------------------------------------------------------------------------------------------------------

\def\newblock#1#2#3{\expandafter\def\csname Block#1\endcsname{\begin{block}{#2}\rmfamily{#3}\end{block}}}
\def\newAlert#1#2#3{\expandafter\def\csname Alert#1\endcsname{\begin{alertblock}{#2}\rmfamily{#3}\end{alertblock}}}
\def\newSingle#1#2{\expandafter\def\csname Single#1\endcsname{\begin{column}{\onecolwid}#2\end{column}}}
\def\newMarried#1#2{\expandafter\def\csname Married#1\endcsname{\begin{column}{\twocolwid}#2\end{column}}}
\def\newTwin#1#2{\expandafter\def\csname Twin#1\endcsname{\begin{columns}[t,totalwidth=\twocolwid]#2\end{columns}}}





%-----------------------------------------------------------
% Define the column width and poster size
% To set effective sepwid, onecolwid and twocolwid values, first choose how many columns you want and how much separation you want between columns
% The separation I chose is 0.024 and I want 4 columns
% Then set onecolwid to be (1-(4+1)*0.024)/4 = 0.22
% Set twocolwid to be 2*onecolwid + sepwid = 0.464
%-----------------------------------------------------------

\newlength{\sepwid}
\newlength{\onecolwid}
\newlength{\twocolwid}
\setlength{\paperwidth}{48in}
\setlength{\paperheight}{36in}
\setlength{\sepwid}{0.024\paperwidth}
\setlength{\onecolwid}{0.22\paperwidth}
\setlength{\twocolwid}{0.464\paperwidth}
\setlength{\topmargin}{-0.5in}
\usetheme{confposter}

%-----------------------------------------------------------
% Define colours (see beamerthemeconfposter.sty to change these colour definitions)
%-----------------------------------------------------------

\setbeamercolor{block title}{fg=ngreen,bg=white}
\setbeamercolor{block body}{fg=black,bg=white}
\setbeamercolor{block alerted title}{fg=white,bg=dblue!70}
\setbeamercolor{block alerted body}{fg=black,bg=dblue!10}






%-----------------------------------------------------------------------------------------------------------------------------------------
%                               START TYPING YOUR BLOCKS !!
%-----------------------------------------------------------------------------------------------------------------------------------------
%%%%%%%%%%%%%%%%%%%%%%%%%%%%%%%%%%%%%%%%%%%%%  IMPORTANT %%%%%%%%%%%%%%%%%%%%%%%%%%%%%%%%%%%%%%%%%%%%%%%%%%%%%%%%%%%%%%%%%%%%%%%%%%%%%%%%%
%%%%%%%%%%%%%%%%%%%%%%%%%%%%%%%%%%%%%%%%%%%%%  IMPORTANT %%%%%%%%%%%%%%%%%%%%%%%%%%%%%%%%%%%%%%%%%%%%%%%%%%%%%%%%%%%%%%%%%%%%%%%%%%%%%%%%%
%%%   NO EMPTY ROWS. This example shows what you are NOT to do:
%%%  \newAlert{Main}{Main First Example}{
%%%             This is the main topic, blah blah blah
%%%
%%%             and then blah blah blah
%%%             }
%%%%%%%%%%%%%%%%%%%%%%%%%%%%%%%%%%%%%%%%%%%%%%%%%%%%%%%%%%%%%%%%%%%%%%%%%%%%%%%%%%%%%%%%%%%%%%%%%%%%%%%%%%%%%%%%%%%%%%%%%%%%%%%%%%%%%%%%%%
















%%%%%%%%%%%%%%%%%%%%%%%%%%%%%%%%%%%%%%%%%%%%%%%%%%%%%%%%%%%%%%%%%%%%%%%%%%%%%%%%%%%%%%%%%%%%%%%%%%%%%%%%%%%%%%%%%%%%%%%%%%%%%%%%%%%%%%%%%%%
%%%%%%%%%%%%%%%%%%%%%%%%%%%%%%%%%%%%%%%%%%%%%%%%%%%%%%%%%%%%%%%%%%%%%%%%%%%%%%%%%%%%%%%%%%%%%%%%%%%%%%%%%%%%%%%%%%%%%%%%%%%%%%%%%%%%%%%%%%%
%%%%%%%%%%%%%%%%%%%%%%%%%%%%%%%%%%%%%%%%%%%%             REGULAR    BLOCKS  START          %%%%%%%%%%%%%%%%%%%%%%%%%%%%%%%%%%%%%%%%%%%%%%%%
%%%%%%%%%%%%%%%%%%%%%%%%%%%%%%%%%%%%%%%%%%%%%%%%%%%%%%%%%%%%%%%%%%%%%%%%%%%%%%%%%%%%%%%%%%%%%%%%%%%%%%%%%%%%%%%%%%%%%%%%%%%%%%%%%%%%%%%%%%%
%%%%%%%%%%%%%%%%%%%%%%%%%%%%%%%%%%%%%%%%%%%%%%%%%%%%%%%%%%%%%%%%%%%%%%%%%%%%%%%%%%%%%%%%%%%%%%%%%%%%%%%%%%%%%%%%%%%%%%%%%%%%%%%%%%%%%%%%%%%




%%%%%%%%%%%%%%%%%%%%%%%%%%%%%%%%%%%%%%%%%%%%%%%%%%%%%%%%%%%%%%%%%%%%%%%%%%%%%%%%%%%%%%%%%%%%%%%%%%%%%%%%%%%%%%%%%%%%%%%%%%%%%%%%%%%%%%%%%%%
\newblock{Introduction}{Introduction}{
        The purpose of our research is to study distributed search-and-escape algorithms.
        These algorithms involve mobile agents (or bots) searching in geometric domains, such as a closed disk or a convex polygon.
        By working together and communicating with one another, the mobile agents search for an exit hidden on the perimeter.
        The goal of our research is to create and study exit strategies that terminate as quickly as possible.
        }
%%%%%%%%%%%%%%%%%%%%%%%%%%%%%%%%%%%%%%%%%%%%%%%%%%%%%%%%%%%%%%%%%%%%%%%%%%%%%%%%%%%%%%%%%%%%%%%%%%%%%%%%%%%%%%%%%%%%%%%%%%%%%%%%%%%%%%%%%%%








%%%%%%%%%%%%%%%%%%%%%%%%%%%%%%%%%%%%%%%%%%%%%%%%%%%%%%%%%%%%%%%%%%%%%%%%%%%%%%%%%%%%%%%%%%%%%%%%%%%%%%%%%%%%%%%%%%%%%%%%%%%%%%%%%%%%%%%%%%%
\newblock{Definitions}{Definitions}{
         An \textbf{exit} is a point unknown to the agents, that is located on an perimeter on the domain.
         The agents must find the exit for the algorithm to finish.
         A  \textbf{priority} agent is one that must reach the exit for the algorithm to terminate.
         A \textbf{helper} agent is one that simply assists the priority agent(s) in finding the exit.
         The exit is \textbf{found} if an agent's coordinates match the exit location.
         The algorithm \textbf{terminates} if a specified subset of agents shares a position with the exit.
         }
%%%%%%%%%%%%%%%%%%%%%%%%%%%%%%%%%%%%%%%%%%%%%%%%%%%%%%%%%%%%%%%%%%%%%%%%%%%%%%%%%%%%%%%%%%%%%%%%%%%%%%%%%%%%%%%%%%%%%%%%%%%%%%%%%%%%%%%%%%%








%%%%%%%%%%%%%%%%%%%%%%%%%%%%%%%%%%%%%%%%%%%%%%%%%%%%%%%%%%%%%%%%%%%%%%%%%%%%%%%%%%%%%%%%%%%%%%%%%%%%%%%%%%%%%%%%%%%%%%%%%%%%%%%%%%%%%%%%%%%
\newblock{Description}{Algorithm Description}{
         In our algorithm, we use two \textbf{priority} agents and one \textbf{helper} agent.
         The termination condition in our algorithm is reached when either one of the two priority agents reaches the exit.
         % INCLUDE A FRAME OF REFERENCE. 0 DEGREES IS AT 3:00. %
         We send one priority and one helper to some angle $\alpha$ on the perimeter of the shape, in the third quadrant.
         The priority agent travels counter-clockwise, and the helper travels clockwise.
         The other priority agent travels to 0 radians and travels counter-clockwise.
         % INCLUDE A FRAME OF REFERENCE. 0 DEGREES IS AT 3:00. %
         }
%%%%%%%%%%%%%%%%%%%%%%%%%%%%%%%%%%%%%%%%%%%%%%%%%%%%%%%%%%%%%%%%%%%%%%%%%%%%%%%%%%%%%%%%%%%%%%%%%%%%%%%%%%%%%%%%%%%%%%%%%%%%%%%%%%%%%%%%%%%

%	the exit is unknown, and that this is an animation of a distributed algorithm,
%	and in the is distibuted algoirithm, we have a multiple bots trying to find an  exit,
%	where two are distinguished and one is a helper, where only one distinguish must reach.
%	crab.rutgers.edu/~shende/ rutgers library system first and search for them, or look through articles, by a link that says do you want an onlune version, which will give access to al l the source, as in the pdf, but not directly.
%	2018 God save the queen
%	fun with algorithms - god saves the queen
%




%%%%%%%%%%%%%%%%%%%%%%%%%%%%%%%%%%%%%%%%%%%%%%%%%%%%%%%%%%%%%%%%%%%%%%%%%%%%%%%%%%%%%%%%%%%%%%%%%%%%%%%%%%%%%%%%%%%%%%%%%%%%%%%%%%%%%%%%%%%
\newblock{Work}{Future Work}{
	    Later on we will study further distributed algorithms of search and escape, such as lines or triangles.
        }
%%%%%%%%%%%%%%%%%%%%%%%%%%%%%%%%%%%%%%%%%%%%%%%%%%%%%%%%%%%%%%%%%%%%%%%%%%%%%%%%%%%%%%%%%%%%%%%%%%%%%%%%%%%%%%%%%%%%%%%%%%%%%%%%%%%%%%%%%%%










%%%%%%%%%%%%%%%%%%%%%%%%%%%%%%%%%%%%%%%%%%%%%%%%%%%%%%%%%%%%%%%%%%%%%%%%%%%%%%%%%%%%%%%%%%%%%%%%%%%%%%%%%%%%%%%%%%%%%%%%%%%%%%%%%%%%%%%%%%%
\newblock{Conclusions}{Conclusions}{
	This algorithm has an upper bound of blank, being faster than an algorithm of one distinguished and two helpers with an upper bound of 3.83.
             \begin{center}
                 \includegraphics[width=8in]{mypics/temp.jpg}\\[.2cm]
                 \captione{10in}{Temp 4}
             \end{center}
       }
%%%%%%%%%%%%%%%%%%%%%%%%%%%%%%%%%%%%%%%%%%%%%%%%%%%%%%%%%%%%%%%%%%%%%%%%%%%%%%%%%%%%%%%%%%%%%%%%%%%%%%%%%%%%%%%%%%%%%%%%%%%%%%%%%%%%%%%%%%%

















%%%%%%%%%%%%%%%%%%%%%%%%%%%%%%%%%%%%%%%%%%%%%%%%%%%%%%%%%%%%%%%%%%%%%%%%%%%%%%%%%%%%%%%%%%%%%%%%%%%%%%%%%%%%%%%%%%%%%%%%%%%%%%%%%%%%%%%%%%%
\newblock{Bibliography}{References}{\small
      \begin{thebibliography}{99}
         \bibitem{JLTED 2014} Jurek Czyzowicz,  et. al., Evacuating Robots via Unknown Exit in a Disk, Springer-Verlag Berlin Heidelberg; 2014.
         \bibitem{JKREDLJS 2018} Jurek Czyzowicz, et. al., Fun with Algorithms, arXiv:1804.06011v1 [cs.MA] 17 Apr 2018.
      \end{thebibliography}
             \begin{center}
                 \includegraphics[width=8in]{mypics/temp.jpg}\\[.2cm]
                 \captione{10in}{Temp 5}
             \end{center}
      }
%%%%%%%%%%%%%%%%%%%%%%%%%%%%%%%%%%%%%%%%%%%%%%%%%%%%%%%%%%%%%%%%%%%%%%%%%%%%%%%%%%%%%%%%%%%%%%%%%%%%%%%%%%%%%%%%%%%%%%%%%%%%%%%%%%%%%%%%%%%













%%%%%%%%%%%%%%%%%%%%%%%%%%%%%%%%%%%%%%%%%%%%%%%%%%%%%%%%%%%%%%%%%%%%%%%%%%%%%%%%%%%%%%%%%%%%%%%%%%%%%%%%%%%%%%%%%%%%%%%%%%%%%%%%%%%%%%%%%%%
\newblock{Acknowledgements}{Acknowledgements}{\small
      This research is supported by the National Science Foundation under grant \# CCf-AF 1813940 (RUI: Search, Evacuation and Reconfiguration with Coordinated Mobile Agents).
      }
%%%%%%%%%%%%%%%%%%%%%%%%%%%%%%%%%%%%%%%%%%%%%%%%%%%%%%%%%%%%%%%%%%%%%%%%%%%%%%%%%%%%%%%%%%%%%%%%%%%%%%%%%%%%%%%%%%%%%%%%%%%%%%%%%%%%%%%%%%%










%%%%%%%%%%%%%%%%%%%%%%%%%%%%%%%%%%%%%%%%%%%%%%%%%%%%%%%%%%%%%%%%%%%%%%%%%%%%%%%%%%%%%%%%%%%%%%%%%%%%%%%%%%%%%%%%%%%%%%%%%%%%%%%%%%%%%%%%%%%
\newblock{Logo}{}{\vspace{-4cm}
      \begin{center}
          \includegraphics[width=1.5in]{mypics/temp.jpg}
      \end{center}
      }
%%%%%%%%%%%%%%%%%%%%%%%%%%%%%%%%%%%%%%%%%%%%%%%%%%%%%%%%%%%%%%%%%%%%%%%%%%%%%%%%%%%%%%%%%%%%%%%%%%%%%%%%%%%%%%%%%%%%%%%%%%%%%%%%%%%%%%%%%%%










%%%%%%%%%%%%%%%%%%%%%%%%%%%%%%%%%%%%%%%%%%%%%%%%%%%%%%%%%%%%%%%%%%%%%%%%%%%%%%%%%%%%%%%%%%%%%%%%%%%%%%%%%%%%%%%%%%%%%%%%%%%%%%%%%%%%%%%%%%%
%%%%%%%%%%%%%%%%%%%%%%%%%%%%%%%%%%%          REGULAR   BLOCKS     FINISH                   %%%%%%%%%%%%%%%%%%%%%%%%%%%%%%%%%%%%%%%%%%%%%%%%
%%%%%%%%%%%%%%%%%%%%%%%%%%%%%%%%%%%%%%%%%%%%%%%%%%%%%%%%%%%%%%%%%%%%%%%%%%%%%%%%%%%%%%%%%%%%%%%%%%%%%%%%%%%%%%%%%%%%%%%%%%%%%%%%%%%%%%%%%%%
















%%%%%%%%%%%%%%%%%%%%%%%%%%%%%%%%%%%%%%%%%%%%%%%%%%%%%%%%%%%%%%%%%%%%%%%%%%%%%%%%%%%%%%%%%%%%%%%%%%%%%%%%%%%%%%%%%%%%%%%%%%%%%%%%%%%%%%%%%%%
%%%%%%%%%%%%%%%%%%%%%%%%%%%%%%%%%%%%%%%%%%%%%%%%%%%%%%%%%%%%%%%%%%%%%%%%%%%%%%%%%%%%%%%%%%%%%%%%%%%%%%%%%%%%%%%%%%%%%%%%%%%%%%%%%%%%%%%%%%%
%%%%%%%%%%%%%%%%%%%%%%%%%%%%%%                  ALERT BLOCKS START                         %%%%%%%%%%%%%%%%%%%%%%%%%%%%%%%%%%%%%%%%%%%%%%%%
%%%%%%%%%%%%%%%%%%%%%%%%%%%%%%%%%%%%%%%%%%%%%%%%%%%%%%%%%%%%%%%%%%%%%%%%%%%%%%%%%%%%%%%%%%%%%%%%%%%%%%%%%%%%%%%%%%%%%%%%%%%%%%%%%%%%%%%%%%%
%%%%%%%%%%%%%%%%%%%%%%%%%%%%%%%%%%%%%%%%%%%%%%%%%%%%%%%%%%%%%%%%%%%%%%%%%%%%%%%%%%%%%%%%%%%%%%%%%%%%%%%%%%%%%%%%%%%%%%%%%%%%%%%%%%%%%%%%%%%


\newAlert{MainTopic}{Main Topic}{
                     \begin{center}
                         \includegraphics[width=2.5in]{mypics/temp.jpg}\hspace{1cm}\includegraphics[width=7in]{mypics/temp.jpg}\hspace{1cm}  \\[.2cm]
                         \captione{15in}{Temp 2}
                     \end{center}
                     \begin{center}
                         \includegraphics[width=5in]{mypics/temp.jpg}\hspace{1cm}\includegraphics[width=7in]{mypics/temp.jpg}\hspace{1cm}  \\[.2cm]
                         \captione{15in}{Temp 3}
                     \end{center}
            }

\newAlert{MainProblem}{Main Problem(Proposition)}{
            The purpose of our research is to propose an algorithm for two distinguished and a helper bot to find an unknown exit on a disk.
%             \begin{thm}
%                $$\tau \in G^\sigma \Leftrightarrow \sigma \in G^\tau$$
%             \end{thm}
             \begin{center}
                 \includegraphics[width=8in]{mypics/temp.JPG}\\[.2cm]
                 \captione{8in}{Temp 1}
             \end{center}
            }

\newAlert{Conjectures}{Our Conjectures and Analysis}{
            \begin{tabular}{lr}
              \conjetura{This algorithm does not have the best lower bound.}{True}
              \conjetura{If the game is linear with $n$ squares then its graph is a path with $n$ vertices. }{True}
              \conjetura {If a game is non-linear its graph must have a vertex degree 2 or more. }{True}
              \conjetura{If a game is non-linear its graph must have at least one vertex with degree three.}{False}
              \conjetura{There is a non-linear game whose graph is almost a path.}{True}
              \conjetura{Any game with 3 or less squares is linear.}{True}
            \end{tabular}
            }


%%%%%%%%%%%%%%%%%%%%%%%%%%%%%%%%%%%%%%%%%%%%%%%%%%%%%%%%%%%%%%%%%%%%%%%%%%%%%%%%%%%%%%%%%%%%%%%%%%%%%%%%%%%%%%%%%%%%%%%%%%%%%%%%%%%%%%%%%%%
%%%%%%%%%%%%%%%%%%%%%%%%%%%%%%%%%%%         ALERT BLOCKS  FINISH                           %%%%%%%%%%%%%%%%%%%%%%%%%%%%%%%%%%%%%%%%%%%%%%%%
%%%%%%%%%%%%%%%%%%%%%%%%%%%%%%%%%%%%%%%%%%%%%%%%%%%%%%%%%%%%%%%%%%%%%%%%%%%%%%%%%%%%%%%%%%%%%%%%%%%%%%%%%%%%%%%%%%%%%%%%%%%%%%%%%%%%%%%%%%%









%%%%%%%%%%%%%%%%%%%%%%%%%%%%%%%%%%%%%%%%%%%%%%   TIPS  TO MAKE A NICE POSTER  %%%%%%%%%%%%%%%%%%%%%%%%%%%%%%%%%%%%%%%%%%%%%%%%%%%%%%%%%%%%%
% Common mistakes when compiling
% 1) There is an empty row in one of the blocks, remove it.
% 2) You forgot to close the brace } at the end of the block, close it.
% 3) The name of a block does not agree, for example you have \newSingle{Description}{\BlockDescription}
%                           and when you call it to form the main columns, you may have typed \SingleDescripption
%                           hence you should have typed  \SingleDescription, correct it.
% 4) It can not find a picture, perhaps you have it in the wrong folder.
% 5) Maybe it doesn't find mypostersettings.tex, it should be in the same folder as CoolPoster.tex
%
%
%%%%%  General Recomendations when typing your poster:
%%%%%  Your regular blocks MUST be short, two or three sentences each, see the examples above.
%%%%%  If your graph contains graphs, first genererate the pdf versions of each graph via LATEX,  then DVItoPS then PStoPDF
%%%%%                   as shown in the tex files contained in the folder mypics
%%%%%%%%%%%%%%%%%%%%%%%%%%%%%%%%%%%%%%%%%%%%%%%%%%%%%%%%   END OF TIPS  %%%%%%%%%%%%%%%%%%%%%%%%%%%%%%%%%%%%%%%%%%%%%%%%%%%%%%%%%%%%%%%%%%
