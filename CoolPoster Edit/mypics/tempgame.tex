%%%%%%%%%%%%%%%%%%%%%%%%%%%%%%%%%%%%%%%%%%%%%%%%%%%%%%%%%%%%%%%%%%%%%%%%%%%%%%%%%%%%%%%%%%%%%
% Compiling this file via: LATEX,  then DVItoPS then PStoPDF will produce a nice pdf file
% which you can include it later as a picture in a file with the \includegraphics...
%%%%%%%%%%%%%%%%%%%%%%%%%%%%%%%%%%%%%%%%%%%%%%%%%%%%%%%%%%%%%%%%%%%%%%%%%%%%%%%%%%%%%%%%%%%%%

\documentclass[12pt]{article}

\usepackage{ps4pdf,pstricks,pstricks-add}

\begin{document}

   \PSforPDF{\psset{unit=1cm}
              \begin{pspicture}(-0.7,-.5)(2.5,4.5)
                \psset{gridcolor=green,subgridcolor=yellow}
                       \psgrid[gridlabels=0pt]
                       \let\psgrid\relax
                       \psline[linewidth=1pt]{-}(0,4)(1,4)
                       \psline[linewidth=1pt]{-}(1,4)(1,1)
                       \psline[linewidth=1pt]{-}(0,4)(0,1)
                       \psline[linewidth=1pt]{-}(0,1)(2,1)
                       \psline[linewidth=1pt]{-}(0,2)(2,2)
                       \psline[linewidth=1pt]{-}(0,3)(2,3)
                       \psline[linewidth=1pt]{-}(2,3)(2,0)
                       \psline[linewidth=1pt]{-}(1,0)(2,0)
                       \psline[linewidth=1pt]{-}(1,1)(1,0)
%                       \rput[t](.5,3.75){\Large \textbf{1}}
%                       \rput[t](.5,2.75){\Large \textbf{2}}
%                       \rput[t](.5,1.75){\Large \textbf{4}}
%                       \rput[t](1.5,2.75){\Large \textbf{3}}
%                       \rput[t](1.5,1.5){\Large \textbf{5}}
%                       \rput[t](1.5,.65){\Large \textbf{*}}
              \end{pspicture}
              }

\end{document} 