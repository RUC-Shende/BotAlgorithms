\documentclass[11pt]{amsart}
\usepackage[margin=2.5cm]{geometry}                % See geometry.pdf to learn the layout options. There are lots.
\geometry{letterpaper}                   % ... or a4paper or a5paper or ...
%\geometry{landscape}                % Activate for for rotated page geometry
%\usepackage[parfill]{parskip}    % Activate to begin paragraphs with an empty line rather than an indent
\usepackage{algorithm}
\usepackage[noend]{algpseudocode}
\usepackage{graphicx}
\usepackage{amssymb,amsfonts}
\usepackage{epstopdf}
\usepackage{multicol}
\DeclareGraphicsRule{.tif}{png}{.png}{`convert #1 `dirname #1`/`basename #1 .tif`.png}

\title{Brief Article}
\author{The Author}
%\date{}                                           % Activate to display a given date or no date

\begin{document}
\maketitle
\section{Preliminaries}

Algorithms of search-and-escape involve mobile agents (also called Robots)
searching in geometric domains, such as a closed disk, or convex polygon. By
working together and communicating with one another, these mobile agents search
the domain to find an exit hidden on the perimeter. Many different problems exist for
this topic, such as evacuating all agents, or only evacuating a specific subset of these
agents.

\subsection{Model}

All agents in this problem use the same coordinate system and operate in a closed
disk, all starting from the center. These agents' algorithms do not all have to be the same,
in fact to most efficiently search the circle, they must all be unique.
In our problem, we observe the Priority model of algorithms. In this model, a
subset of one or more agents (P) is defined as a Priority (or Queen) and the goal
of the algorithm is to evacuate a certain number of these Priority Agents. These
algorithms also include a certain number Helper agents (H), that simply assist in searching the circle
for the exit, for a total of (H + P) agents. The Helper agents are not typically required to evacuate.
Once an exit is found, whether by a Helper or a Priority,the agent may use
Wireless communication to immediately broadcast the exit's location and the finder's identity to all other agents.
Upon receiving this broadcasted location, any remaining Priority agents that
need to evacuate travel immediately to the exit and the algorithm terminates.
The cost of the algorithm is called the termination time, and is the total worst-case
time for the required subset of Priority agents to exit.

\subsection{Previous Work}

Previously, problems of a similar type have been studied, namely those regarding 1 Priority and 1
or more Helper agents searching in a closed disk. [(God Save the Queen, 2018)
(Priority Evacuation From a Disk Using Mobile Robots, 2018)]
In these papers, the results involved getting the only Priority agent to the exit
as fast as possible, however, our problem attempts to design an algorithm where
only one of multiple Priority agents needs to evacuate.


\subsection{Our Results}

In our algorithm containing 2 Priority and 1 Helper agent, we show that a termination time
upper bound of 3.55 time units is possible given the specific set of parameters we use
to guide the agents.




\section{Algorithm 1}

%Our first priority algorithm with 2 queens and 1 servant

\begin{algorithm}
  \caption{Priority and Helper Algorithm}
  \begin{algorithmic}[1]
    \Procedure{Search}{$\alpha$}\Comment{Search for exit and evacuate closest Priority}
      \State $Q1, Q2, H$ are 2 Priority and Helper respectively.
      \State All angles are on the typical unit circle.
      \State $Q1$ goes to the perimeter of the disk at angle 0.
      \State $Q2$ and $H$ go to the perimeter at angle $\alpha$.
      \Repeat $Q1$ and $H$ travel clockwise and $Q2$ travels counterclockwise \Until{Exit is found.}
      \If {$H$ Finds the exit} $H$ broadcasts the exit location, and the closer of the
      two ($Q1$ or $Q2$) travels along a chord directly to the exit. The algorithm then terminates. \EndIf
      \If {$Q1$ or $Q2$ finds the exit} the algorithm terminates. \EndIf
    \EndProcedure
  \end{algorithmic}
\end{algorithm}





\end{document}
